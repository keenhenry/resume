%______________________________________________________________________________
% Henry Huang's resume using Kamil K Wojcicki's template
% @brief    LaTeX2e Resume for Kamil K Wojcicki
\documentclass[margin,line]{resume}

\usepackage{url}

%_______________________________________________________________________________
\begin{document}
\name{\Large Hsiu-Yu Huang \hspace{82 mm}}
\begin{resume}

    %___________________________________________________________________________
    % Contact Information
    \section{\myheadingstyle Contact \\ Information}
      \begin{tabular}{@{}p{7cm}p{6cm}}
      \url{keenhenry.me}      & Parkzichtlaan 238A \\
      keenhenry1109@gmail.com & 3544 MN, Utrecht \\
      +31-614-509-054        & The Netherlands \\
      \end{tabular}
 
    %___________________________________________________________________________
    % Summary
    \section{\myheadingstyle Summary}
      % this parts has to be modified along time
      - Experienced Data Engineer \\
      - Strong interests in Data Science and Big Data (using Python) \\
      - Full stack web development (using Python in the backend) \\
      - Passionate about using Python for automation \\
      - Geo spatial data processing using SQLite/Spatialite GIS \\
    \vspace{-5mm}

    %__________________________________________________________________________
    % Work Experience
    \section{\myheadingstyle Work \\ Experience}
    \textbf{FRISS Fraudebestrijding B.V.}, Utrecht, The Netherlands \\
    \textsl{Data Engineer / Data Scientist} \hfill 11/2017 -- Present \vspace{-3mm}\\\vspace{-1mm}%
      \begin{list2}
      \item Introduced proper code review practice
      \item Automated creation and provision of development environment with Vagrant and Ansible
      \item Developed and maintained R applications
      \item Maintained and improved ETL python codebase
      \item Technologies used: MicroSoft Team Foundation Server, Git, Python, Pytest, R, Bash, SQL, R Shiny, Vagrant, Ansible, Docker,
            ElasticSearch, PostgreSQL, MongoDB, Neo4j, Linux Ubuntu, Hyper-V, OpenCPU
      \end{list2}

    \textbf{ASML Netherlands B.V.}, Eindhoven, The Netherlands \\
    \textsl{Software Development Engineer Consultant from MCA Nederland B.V.} \hfill 10/2016 -- 10/2017 \vspace{-3mm}\\\vspace{-1mm}%
      \begin{list2}
      \item Senior software developer in a team of four software engineers
      \item Designed and architected a complete data warehouse solution plus two web systems
      \item Full stack web development: front/back-end, automated tests, delivery and deployment, \newline system administration
      \item Introduced Jira and Bitbucket into the team
      \item Advocated and introduced proper code review practice
      \item Implemented Git repository dependency management with Git Subtree
      \item Helped the team set up development environment (CentOS with VirtualBox)
      \item Helped the team set up Continuous Integration System (Jenkins)
      \item Implemented Continuous Delivery with Python Fabric and Jenkins
      \item Technologies used: Jira, Bitbucket, Git, Python, Gunicorn, Django, Circus, Supervisord, Huey, Celery, Skeleton CSS, jQuery,
            Bash, SQL, Jenkins, SQLite, Redis, RabbitMQ, Nginx, CentOS, Redhat, VirtualBox, Vagrant, System V Init, Upstart Init
      \end{list2}

    \textbf{Mapscape B.V.}, Eindhoven, The Netherlands \\
    \textsl{Software Engineer} \hfill 10/2011 -- 09/2016 \vspace{-3mm}\\\vspace{-1mm}%
      \begin{list2}
      \item Designed, developed and maintained software to extract, transform and store map data
      \item Performed data analysis and create simple statistics by using SQL, Python or Bash scripts
      \item Created tools in Python to automate recurring tasks for development and testing
      \item Agile Development using Kanban
      \item Mentored new colleagues and shared management role of projects
      \item Technologies used: C/C++, Boost, Git, Jira, Jenkins, Python, Bash, SQL, SQLite/Spatialite
      \end{list2}

    %___________________________________________________________________________
    % Education
    \section{\myheadingstyle Education}

    \textbf{Columbia University}, New York, NY \hfill 09/2008 -- 05/2010 \vspace{-3mm}\\\vspace{-1mm}%
      \begin{list2}
       \item M.S. in Computer Science
      \end{list2}
    \vspace{-1mm}
 
    %___________________________________________________________________________
    % Online Courses: Self-learning
    \section{\myheadingstyle Online \\ Courses}

    \textbf{Machine Learning A-Z (Udemy)} \hfill 11/2017 -- Present \vspace{-3mm}\\\vspace{-1mm}%
      \begin{list2}
       \item Learning various Machine Learning techniques and algorithms
       \item Implement those machine learning techniques with both Python and R code
      \end{list2}
    \vspace{-2mm}

    % \textbf{Intro to Data Analysis} \hfill 07/2016 -- Present \vspace{-3mm}\\\vspace{-1mm}%
    %   \begin{list2}
    %    \item Learn to do data analysis and visualization by using Python pandas and numpy libraries
    %   \end{list2}
    % \vspace{-2mm}

    \textbf{Intro to Hadoop and MapReduce (Udacity)} \hfill 07/2016 -- 01/2017 \vspace{-3mm}\\\vspace{-1mm}%
      \begin{list2}
       \item Learned the definitions of Big Data and the patterns of Big Data problems
       \item Learned the basics of Hadoop ecosystem with a focus on MapReduce framework and HDFS
       \item Learned to write MapReduce programs with Hadoop Streaming to solve Big Data problems
      \end{list2}
    \vspace{-2mm}
    % \textbf{Full Stack Foundations} \url{https://www.udacity.com/course/ud088} \hfill 02/2015 -- 04/2015 \vspace{-3mm}\\\vspace{-1mm}%
    %   \begin{list2}
    %    \item Learned to use Flask and SQLAlchemy to build a web application
    %   \end{list2}
    % \vspace{-2mm}
    %
    % \textbf{Object-Oriented Javascript} \url{http://bit.ly/1GVqF3g} \hfill 04/2015 \vspace{-3mm}\\\vspace{-1mm}%
    %   \begin{list2}
    %    \item Learned Javascript OO patterns to write reusable code
    %   \end{list2}
    % \vspace{-2mm}
    % \textbf{Web Development} \url{https://www.udacity.com/course/cs253} \hfill 10/2014 -- 01/2015 \vspace{-3mm}\\\vspace{-1mm}%
    %   \begin{list2}
    %    \item Learned Python back-end web development using Google App Engine
    %   \end{list2}
    % \vspace{-2mm}
    % can be commented out for Backend developer position
    % \textbf{Javascript Basics} \url{https://www.udacity.com/course/ud804} \hfill 02/2015 \vspace{-3mm}\\\vspace{-1mm}%
    %   \begin{list2}
    %    \item Learned to use basic Javascript data structures and program flow control
    %   \end{list2}
    % \vspace{-2mm}
    % \textbf{Intro to jQuery and AJAX} \url{https://www.udacity.com/course/nd001} \hfill 02,03/2015 \vspace{-3mm}\\\vspace{-1mm}%
    %   \begin{list2}
    %    \item Learned to use jQuery for DOM selection and manipulation
    %    \item Learned to use jQuery to send AJAX requests in normal and JSONP formats
    %   \end{list2}
    % \vspace{-2mm}
    % \textbf{Website Performance Optimization} \url{http://bit.ly/1XLxEq4} \hfill 09/2015 \vspace{-3mm}\\\vspace{-1mm}%
    %   \begin{list2}
    %    \item Applied website performance improvement techniques to personal wiki site
    %   \end{list2}
    % \vspace{-2mm}

    %___________________________________________________________________________
    % Project Experience
    \newpage

    \section{\myheadingstyle Projects}

    % describe the internal project experience you have for automating tasks
    \textbf{MapFlow} \hfill 11/2015 \vspace{-3mm}\\\vspace{-1mm}%
      \begin{list2}
       \item A self-initiative project aimed to improve map creation work-flow within Mapscape
       \item A file system crawler built for collecting map meta-data
       \item A tool for visualizing simple statistical results of crawled meta-data created
       \item A few small services created to improve working efficiency by using map metadata
       \item Using Airflow platform to automate map creation pipeline (not completed)
       \item Technologies used: Python multiprocessing, argparse, pandas, matplotlib, SQLite, Parsley, Redis
      \end{list2}


    \textbf{Personal Wiki} - \url{http://keenhenry-wiki.appspot.com} \hfill 01/2015 \vspace{-3mm}\\\vspace{-1mm}%
      \begin{list2}
       \item A personal wiki hosted on Google App Engine
       \item Support editing in Markdown syntax
       \item Incorporated a user registration/authentication system
       \item Page editing history is stored
       \item Used Selenium for browser testing automation
       %\item Implemented Full Text Search
       \item Technologies used: Python, webapp2, jinja2, CSS3, HTML5, jQuery, Bootstrap 3, memcache, Selenium
      \end{list2}

    % can be commented out for non-web position
    % \textbf{eServ} - \url{https://github.com/keenhenry/eserv} \hfill 12/2014 \vspace{-3mm}\\\vspace{-1mm}%
    %   \begin{list2}
    %    \item A multi-threaded HTTP server for embedded devices
    %    \item Support dynamic content generation
    %    \item Support session cookie
    %    \item Technologies used: C, pthread, jQuery, AJAX
    %   \end{list2}

    % \textbf{Volkswagen MIB and MIB2 projects in Mapscape} \hfill 01/2013 -- 12/2015 \vspace{-3mm}\\\vspace{-1mm}%
    %   \begin{list2}
    %   \item Improved map data for OpenGL to draw triangles correctly and efficiently
    %   \item Implemented junction view products
    %   \item Repaired visualization defects and improved polygon merging algorithm for elevation contour and world boundary maps
    %   \item Implemented full text search features for POI's data in NDS (Navigation Data Standard) maps
    %   \item Redesigned and implemented connected streets for North America products
    %   \end{list2}

    \textbf{Personal Desktop Assistant} - \url{https://pypi.python.org/pypi/pda/0.3.0.2} \hfill 03/2014 \vspace{-3mm}\\\vspace{-1mm}%
      \begin{list2}
       \item A Python command line to-do list manager
       \item Storing list data both locally and remotely (on Github Issues)
       \item Configurable through a configuration file
       \item Support both Python 2 and 3
       \item Used Travis CI service for code integration
       \item Technologies used: Pylint, Python unittest, nose, tox, Github Issues API, reStructuredText
      \end{list2}

    \textbf{Game Isolate} - \url{http://bit.ly/eiooQ0} \hfill 10/2010 \vspace{-3mm}\\\vspace{-1mm}%
      \begin{list2}
       \item An Android mobile application
       \item Game AI engine implemented using $\alpha \beta$ search algorithm and quiescence search
       \item Technologies used: Java, Android SDK, Ant, Proguard, Multithreading
      \end{list2}

%_______________________________________________________________________________
\end{resume}
\end{document}


%_______________________________________________________________________________
% EOF
